% Options for packages loaded elsewhere
\PassOptionsToPackage{unicode}{hyperref}
\PassOptionsToPackage{hyphens}{url}
\PassOptionsToPackage{dvipsnames,svgnames,x11names}{xcolor}
%
\documentclass[
  letterpaper,
  DIV=11,
  numbers=noendperiod]{scrartcl}

\usepackage{amsmath,amssymb}
\usepackage{iftex}
\ifPDFTeX
  \usepackage[T1]{fontenc}
  \usepackage[utf8]{inputenc}
  \usepackage{textcomp} % provide euro and other symbols
\else % if luatex or xetex
  \usepackage{unicode-math}
  \defaultfontfeatures{Scale=MatchLowercase}
  \defaultfontfeatures[\rmfamily]{Ligatures=TeX,Scale=1}
\fi
\usepackage{lmodern}
\ifPDFTeX\else  
    % xetex/luatex font selection
\fi
% Use upquote if available, for straight quotes in verbatim environments
\IfFileExists{upquote.sty}{\usepackage{upquote}}{}
\IfFileExists{microtype.sty}{% use microtype if available
  \usepackage[]{microtype}
  \UseMicrotypeSet[protrusion]{basicmath} % disable protrusion for tt fonts
}{}
\makeatletter
\@ifundefined{KOMAClassName}{% if non-KOMA class
  \IfFileExists{parskip.sty}{%
    \usepackage{parskip}
  }{% else
    \setlength{\parindent}{0pt}
    \setlength{\parskip}{6pt plus 2pt minus 1pt}}
}{% if KOMA class
  \KOMAoptions{parskip=half}}
\makeatother
\usepackage{xcolor}
\setlength{\emergencystretch}{3em} % prevent overfull lines
\setcounter{secnumdepth}{-\maxdimen} % remove section numbering
% Make \paragraph and \subparagraph free-standing
\ifx\paragraph\undefined\else
  \let\oldparagraph\paragraph
  \renewcommand{\paragraph}[1]{\oldparagraph{#1}\mbox{}}
\fi
\ifx\subparagraph\undefined\else
  \let\oldsubparagraph\subparagraph
  \renewcommand{\subparagraph}[1]{\oldsubparagraph{#1}\mbox{}}
\fi


\providecommand{\tightlist}{%
  \setlength{\itemsep}{0pt}\setlength{\parskip}{0pt}}\usepackage{longtable,booktabs,array}
\usepackage{calc} % for calculating minipage widths
% Correct order of tables after \paragraph or \subparagraph
\usepackage{etoolbox}
\makeatletter
\patchcmd\longtable{\par}{\if@noskipsec\mbox{}\fi\par}{}{}
\makeatother
% Allow footnotes in longtable head/foot
\IfFileExists{footnotehyper.sty}{\usepackage{footnotehyper}}{\usepackage{footnote}}
\makesavenoteenv{longtable}
\usepackage{graphicx}
\makeatletter
\def\maxwidth{\ifdim\Gin@nat@width>\linewidth\linewidth\else\Gin@nat@width\fi}
\def\maxheight{\ifdim\Gin@nat@height>\textheight\textheight\else\Gin@nat@height\fi}
\makeatother
% Scale images if necessary, so that they will not overflow the page
% margins by default, and it is still possible to overwrite the defaults
% using explicit options in \includegraphics[width, height, ...]{}
\setkeys{Gin}{width=\maxwidth,height=\maxheight,keepaspectratio}
% Set default figure placement to htbp
\makeatletter
\def\fps@figure{htbp}
\makeatother

\KOMAoption{captions}{tableheading}
\makeatletter
\@ifpackageloaded{tcolorbox}{}{\usepackage[skins,breakable]{tcolorbox}}
\@ifpackageloaded{fontawesome5}{}{\usepackage{fontawesome5}}
\definecolor{quarto-callout-color}{HTML}{909090}
\definecolor{quarto-callout-note-color}{HTML}{0758E5}
\definecolor{quarto-callout-important-color}{HTML}{CC1914}
\definecolor{quarto-callout-warning-color}{HTML}{EB9113}
\definecolor{quarto-callout-tip-color}{HTML}{00A047}
\definecolor{quarto-callout-caution-color}{HTML}{FC5300}
\definecolor{quarto-callout-color-frame}{HTML}{acacac}
\definecolor{quarto-callout-note-color-frame}{HTML}{4582ec}
\definecolor{quarto-callout-important-color-frame}{HTML}{d9534f}
\definecolor{quarto-callout-warning-color-frame}{HTML}{f0ad4e}
\definecolor{quarto-callout-tip-color-frame}{HTML}{02b875}
\definecolor{quarto-callout-caution-color-frame}{HTML}{fd7e14}
\makeatother
\makeatletter
\makeatother
\makeatletter
\makeatother
\makeatletter
\@ifpackageloaded{caption}{}{\usepackage{caption}}
\AtBeginDocument{%
\ifdefined\contentsname
  \renewcommand*\contentsname{Table of contents}
\else
  \newcommand\contentsname{Table of contents}
\fi
\ifdefined\listfigurename
  \renewcommand*\listfigurename{List of Figures}
\else
  \newcommand\listfigurename{List of Figures}
\fi
\ifdefined\listtablename
  \renewcommand*\listtablename{List of Tables}
\else
  \newcommand\listtablename{List of Tables}
\fi
\ifdefined\figurename
  \renewcommand*\figurename{Figure}
\else
  \newcommand\figurename{Figure}
\fi
\ifdefined\tablename
  \renewcommand*\tablename{Table}
\else
  \newcommand\tablename{Table}
\fi
}
\@ifpackageloaded{float}{}{\usepackage{float}}
\floatstyle{ruled}
\@ifundefined{c@chapter}{\newfloat{codelisting}{h}{lop}}{\newfloat{codelisting}{h}{lop}[chapter]}
\floatname{codelisting}{Listing}
\newcommand*\listoflistings{\listof{codelisting}{List of Listings}}
\makeatother
\makeatletter
\@ifpackageloaded{caption}{}{\usepackage{caption}}
\@ifpackageloaded{subcaption}{}{\usepackage{subcaption}}
\makeatother
\makeatletter
\@ifpackageloaded{tcolorbox}{}{\usepackage[skins,breakable]{tcolorbox}}
\makeatother
\makeatletter
\@ifundefined{shadecolor}{\definecolor{shadecolor}{rgb}{.97, .97, .97}}
\makeatother
\makeatletter
\makeatother
\makeatletter
\makeatother
\ifLuaTeX
  \usepackage{selnolig}  % disable illegal ligatures
\fi
\IfFileExists{bookmark.sty}{\usepackage{bookmark}}{\usepackage{hyperref}}
\IfFileExists{xurl.sty}{\usepackage{xurl}}{} % add URL line breaks if available
\urlstyle{same} % disable monospaced font for URLs
\hypersetup{
  pdftitle={Pre-Analysis Plan},
  pdfauthor={Steven Kerr},
  colorlinks=true,
  linkcolor={blue},
  filecolor={Maroon},
  citecolor={Blue},
  urlcolor={Blue},
  pdfcreator={LaTeX via pandoc}}

\title{Pre-Analysis Plan}
\usepackage{etoolbox}
\makeatletter
\providecommand{\subtitle}[1]{% add subtitle to \maketitle
  \apptocmd{\@title}{\par {\large #1 \par}}{}{}
}
\makeatother
\subtitle{Assessing Public Attention Towards 2022-2023 Mpox Outbreak
Using Wikipedia}
\author{Steven Kerr}
\date{31 January 2024}

\begin{document}
\maketitle
\ifdefined\Shaded\renewenvironment{Shaded}{\begin{tcolorbox}[boxrule=0pt, sharp corners, interior hidden, borderline west={3pt}{0pt}{shadecolor}, breakable, enhanced, frame hidden]}{\end{tcolorbox}}\fi

\hypertarget{summary}{%
\section{Summary}\label{summary}}

This project\footnote{Project GitHub:
  \url{https://github.com/smkerr/Thesis_WorkInProgress}} explores the
potential for Wikipedia data to be utilized as an alternative metric for
gauging public attention towards public health emergencies, specifically
by focusing on the 2022-2023 multi-country mpox outbreak. The primary
goal is to assess whether Wikipedia, with its vast user base and
granular data, can effectively serve as a proxy for measuring public
attention and information-seeking behavior in the context of global
health crises. This work is motivated by the limitations of existing
disease modeling techniques for emerging diseases, which may lack
sufficient historical data for accurate forecasting. Public attention,
as measured by digital resources like Wikipedia, could provide early
indicators of disease outbreaks, thus aiding in more timely and
effective public health interventions.

Data on the number of confirmed mpox cases is obtained from the World
Health Organization (WHO), while page view statistics for Wikipedia
articles related to mpox are sourced directly from Wikipedia. The
analysis is restricted to the 10 countries most affected by the
outbreak, using time series analysis and lag-correlation methods to
examine the relationship between Wikipedia page views and mpox case
numbers. The study aims to identify patterns and correlations that could
support the use of Wikipedia data as a predictive tool for public health
surveillance.

This work contributes to the broader academic discourse by exploring the
potential of open-source data for enhancing disease surveillance and
response strategies. It seeks to validate the efficacy of Wikipedia, as
a non-traditional data source, in providing actionable insights to
policymakers during public health emergencies. In doing so, this work
addresses gaps in the literature regarding the effectiveness of digital
tools in enhancing disease surveillance.

\hypertarget{motivation-background}{%
\section{Motivation \& Background}\label{motivation-background}}

As the world becomes increasingly interconnected and climate change
elevates the risk of zoonotic spillover events, the public becomes ever
more susceptible to global-scale outbreaks.\footnote{\url{https://www.nature.com/articles/s41579-021-00639-z}}
In this context, traditional disease modeling techniques, while
effective for well-documented diseases, may not be as applicable for
emerging diseases with low case numbers and less understood
epidemiological parameters. Critically, public attention plays a pivotal
role in disease detection and response.

The advent of digital tools allows for the real-time tracking of this
attention, offering a valuable complement to established disease
surveillance methods. This is where public attention can serve as an
innovative proxy for tracking disease spread. In this regard, the
2022-2023 multi-country mpox outbreak presents a recent case study,
allowing us to examine the efficacy of real-time data in monitoring a
relatively novel disease. By thoroughly exploring these non-traditional
methods, we can more fully understand their advantages and limitations,
thus providing crucial insights for decision-makers. This research not
only enhances our understanding of digital epidemiology but also
contributes to shaping more effective and timely responses to future
infectious disease outbreaks. For these reasons, I find the research
question to be of particular interest.

This research topic closely connects to my academic and professional
background. Throughout my studies, I have taken opportunities to apply
my data science skills towards projects related to epidemics and
infodemics (primarily COVID-19). During my Professional Year, I worked
with \href{https://cpc-analytics.com/home/}{CPC Analytics}, a
Berlin-based consultancy focused on global health topics, where I
directly contributed to projects for the
\href{https://pandemichub.who.int/}{World Health Organization's (WHO)
Pandemic Hub} and the WHO Health Emergencies Programme's Mpox Data
Analytics Unit. This experience not only provided me with a
comprehensive understanding of the datasets related to mpox but also led
to my contribution to two separate publications:
\href{https://doi.org/10.1016/S2214-109X(23)00198-5}{``Description of
the first global outbreak of mpox: an analysis of global surveillance
data''} and \href{https://doi.org/10.3201/eid2910.230516}{``Mpox in
Children and Adolescents during Multicountry Outbreak, 2022-2023''}.
Currently, I am contributing towards an evidence synthesis of global
health literature at the \href{https://www.mcc-berlin.net/}{Mercator
Research Institute on Global Commons and Climate Change (MCC)}, further
deepening my expertise in epidemiology and fueling my interest in global
health issues. These cumulative experiences have not only expanded my
knowledge in this field but have also solidified my aim to continue
contributing to the filed of global health.

Through this project, I hope to enhance a range of critical skills.
First, this project involves time series analysis as it tracks changes
in public interest and confirmed cases over time, thus deepening my
knowledge of statistical techniques involved in working with time series
data. In terms of domain knowledge, this project requires an in-depth
understanding of public health dynamics, particularly in the context of
disease outbreaks, allowing me to further build on my existing
experience. Lastly, this project will enhance my project management
skills, as it demands careful planning, organization, and time
management to successfully conduct an independent research project
within a set timeframe, thus preparing me for managing future data
science projects effectively.

\hypertarget{introduction}{%
\section{Introduction}\label{introduction}}

Mpox (formerly known as monkeypox) is an infectious disease caused by
the monkeypox virus (MPXV).\footnote{\url{https://www.who.int/health-topics/monkeypox/}}
It was first identified in laboratory monkeys in 1958 and first recorded
in humans in 1970.\footnote{\url{https://www.who.int/europe/health-topics/monkeypox}}
While mpox transmission has been well-documented in parts of western and
central Africa for several decades,\footnote{\url{https://www.who.int/health-topics/monkeypox/}}
beginning in May 2022, a high proportion of cases were reported in
countries which had not previously observed sustained chains of
transmission.\footnote{\url{https://worldhealthorg.shinyapps.io/mpx_global/}}
The unexpected appearance of mpox in several regions in the initial
absence of epidemiological links to areas in western and central Africa,
suggests that there may have been undetected transmission for some time,
especially concerning given that the World Health Organization (WHO)
considers the confirmation of one case of mpox, in a country, to be an
outbreak.\footnote{\url{https://worldhealthorg.shinyapps.io/mpx_global/}}
\footnote{\url{https://www.who.int/emergencies/disease-outbreak-news/item/2022-DON393}}
In light of the rapidly increasing case numbers, WHO declared the
multi-country mpox outbreak to be a Public Health Emergency of
International Concern (PHEIC), the first such declaration since the
COVID-19 Pandemic.\footnote{\url{https://www.who.int/europe/news/item/23-07-2022-who-director-general-declares-the-ongoing-monkeypox-outbreak-a-public-health-event-of-international-concern}}
As of 31 December 2023, 117 WHO Member States across all six WHO regions
have reported mpox cases, amounting to 93,030 confirmed cases, 652
probable cases, and 176 deaths.\footnote{\url{https://worldhealthorg.shinyapps.io/mpx_global/}}

While sophisticated models exist for common diseases such the flu and
COVID-19, outbreaks of relatively novel diseases or large-scale
outbreaks of diseases which have previously had very limited
transmission may lack sufficient prior data on which to build initial
models. This highlights the need for real-time forecasting methods, even
if only in the initial periods until sufficient data are available for
more robust models. Due to its relatively low number of cases and lack
of attention from the global research community prior to the 2022-2023
multi-country outbreak, mpox represents a potential case study to test
whether these sorts of methods might apply to these types of outbreak.
To that end, public attention can play a pivotal role in the detection
and response to infectious diseases, having been used by some scholars
to predict infectious disease outbreaks.\footnote{\url{https://doi.org/10.1371/journal.pcbi.1004239}}
With modern advancements in digital tools, it is now possible to track
this attention in real-time. Understanding real-time health
information-seeking behavior can act as a proxy indicator of public
health needs.

Internet searches have become a critical source of health information,
with Wikipedia representing a widely used online resource for health
information.\footnote{\url{https://doi.org/10.1371/journal.pone.0228786}}
In fact, with regard to medical information, studies have shown that
Wikipedia's popularity exceeds that of the National Health Service,
WebMD, Mayo Clinic, and WHO websites combined.\footnote{\url{https://europepmc.org/article/MED/19390105}}
As such, Wikipedia can be harnessed as a potential tool for infectious
disease outbreak surveillance. Wikipedia page views have already been
used to study and predict the spread of other infectious
diseases.\footnote{\url{https://dx.plos.org/10.1371/journal.pcbi.1004239}}
The analysis of Wikipedia medical article popularity could be a viable
method for epidemiological surveillance, as it provides important
information about the reasons behind public attention and factors that
sustain public interest in the long term.

Several key academic papers lay the groundwork for this approach. In
\href{https://doi.org/10.2196/26331}{``Assessing Public Interest Based
on Wikipedia's Most Visited Medical Articles During the SARS-CoV-2
Outbreak: Search Trends Analysis''} by Chrzanowski et al.~(2021), the
authors investigates how public interest in medical topics changed
during the COVID-19 pandemic, as reflected by Wikipedia pageviews. The
study conducted a retrospective analysis of access to medical articles
across nine language versions of Wikipedia and correlated these patterns
with global and regional COVID-19 deaths, comparing observed data to a
forecast model trained on data from 2015-2019. This involved collecting
daily page view statistics for 37,880 articles curated by the English
Wikipedia Medicine Project from 1 July 2015 to 13 September 2020. The
authors sourced page view statistics using ToolForge, a page view
analytics tool) and sourced COVID-19 death statistics from Our World in
Data. It found a correlation between the pandemic's severity and
pageviews for COVID-19--related Wikipedia articles, concluding that
changes in article popularity could serve as a method for
epidemiological surveillance by reflecting public attention during
disease outbreaks. Furthermore, it demonstrates the potential for using
Wikipedia data for epidemiological surveillance and understanding public
information-seeking behavior during disease outbreaks. While the paper
focuses on the COVID-19 pandemic, similar methodologies can be applied
to provide insights into public attention and information-seeking
behavior during the Mpox outbreak.

\href{https://doi.org/10.1002\%2Fjmv.28382}{``Association between public
attention and monkeypox epidemic: A global lag-correlation analysis''}
by Yan et al.~(2023) investigates the association between public
attention, as measured by Google Trends Index (GTI), and the global mpox
oubtreak. The authors use Google Trends data for information on internet
search activity related to mpox as well as data on daily confirmed mpox
cases from Our World in Data. It tests time-lag correlations between GTI
and daily confirmed mpox cases across the 20 countries with the highest
case numbers as of 20 September 2022 using the Spearman correlation
coefficients, over a range of -36 to +36 days. Spearman correlation
coefficients from these 20 countries were pooled to provide a combined
correlation coefficient for each lag. To test whether the time series
was stationary, an Augment Dickey-Fuller (ADF) test is applied. The
study finds a strong positive correlation, particularly noticeable 13
days after a peak in public attention. The study also conducted
meta-analyses and utilized vector autoregression (VAR) models to analyze
the temporal relationship between GTI and daily confirmed mpox cases,
and a Granger-causality test was employed to evaluate whether the GTI
trend could predict daily confirmed mpox cases. The findings suggest
that GTI could be a useful tool for early monitoring and prediction of
mpox cases, highlighting the significance of digital epidemiology in
infectious disease surveillance. The study emphasizes the potential of
internet data like GTI in providing early warning signs for health
outbreaks and aiding in rapid response strategies. While the study
utilizes GTI data to public attention towards the mpox outbreak, however
similar methods can be applied to Wikipedia pageviews data to test
whether the same conclusions can be drawn using Wikipedia as a data
source.

In \href{https://doi.org/10.3390/ijerph20043395}{``Trends in Online
Search Activity and the Correlation with Daily New Cases of Monkeypox
among 102 Countries or Territories''} by Du et al.~(2023), the authors
investigate the relationship between online search activity related to
mpox and the actual daily new cases of mpox across 102 countries or
territories. The study aims to understand how internet search trends can
reflect public awareness towards mpox, potentially serving as an early
indicator for outbreaks. Data on daily mpox cases from 1 May 2022 to 9
October 2022 was sourced from Our World in Data, while online search
activity data related to mpox was sourced from Google Trends using the
keyword ``monkeypox''.\footnote{Note that this study was conducted prior
  to WHO's recommendation that monkeypox instead be referred to as
  ``mpox'' on 28 November 2022:
  \url{https://www.who.int/news/item/28-11-2022-who-recommends-new-name-for-monkeypox-disease}}
Online search activity was expressed as relative normalized search
volume numbers (RNSNs) ranging from 0 to 100 to reflect how many
searches are performed for a keyword relative to the total number of
searches on the internet over time where a value of 100 represents the
time point at which the search term has reached its peak in popularity.
Demographic data including total population, population density, average
years of schooling, socioeconomic status, and public tourism were
sourced from the United Nations and World Bank. Data on health status
including HIV prevalence, sanitation levels, and health workforce
densities were obtained from the 2019 Global Burden of Disease study.
The authors use a segmented time-series analysis to estimate the impact
of the PHEIC declaration on online search activity, adjusting for daily
new cases across 194 countries or territories. Furthermore, the study
tests time-lag correlations between online search activity and daily new
cases, specifically considering lags of -21, -14, -7, 0, +7, +14, and
+21 days. Next, a general linear regression model (GLM) is used to
explore influencing factors on the relationship between online search
activity and daily new cases. The authors find a significant correlation
between online search activity and daily mpox cases, with online
searches often preceding reporting of new cases. This study highlights
the value of integrating internet search data into public health
surveillance for emerging infectious diseases. Similar to the paper by
Yan et al., this study utilizes Google Trends data, however similar
methodologies could be applied towards Wikipedia pageviews data.

Beyond these studies, several other academic papers have contributed
analysis to various aspects of the relationship between online
information-seeking behavior and public health.

\begin{itemize}
\item
  \href{https://doi.org/10.1098/rsos.160460}{García-Gavilanes et
  al.~(2016)} use Wikipedia page view and page edit statistics to
  investigate public attention towards airline crashes.
\item
  In their analysis of page views for COVID-19-related Wikipedia pages,
  \href{https://doi.org/10.2196/21597}{Gozzi et al.~(2020)} find that
  page views were mainly driven by media coverage, declined rapidly,
  even while COVID-19 incidence remained high, raising questions about
  the impacts of attention saturation in disease outbreak settings.
\item
  \href{https://doi.org/10.3390/info14010005}{Bhagavathula and
  Raubenheimer (2023)} conducted a joinpoint regression analysis to
  measure hourly percentage changes (HPC) in search volume in the hours
  immediately preceding and following WHO's determination to assign
  PHEIC status to mpox, finding an overall increase in
  information-seeking behavior, although results varied by country. This
  study revealed a 103\% increase in public interest in top five
  Mpox-affected countries immediately following the WHO PHEIC
  announcement. However, search interest waned after the announcement,
  so that search interest appeared to reflect media attention more than
  disease spread.
\item
  \href{https://doi.org/10.3389/fpubh.2021.755530}{Gong et al.~(2022)}
  use the Baidu Index (BDI) and Sina Macro Index (SMI) to investigate
  the association between public attention towards the COVID-19 pandemic
  and new cases using Spearman correlation.
\item
  \href{https://doi.org/10.3390/ijerph18094560}{Abbas et al.~(2021)}
  analyze associations between Google Search Trends for symptoms of
  COVID-19 and confirmed cases and deaths within the United States,
  demonstrating abilitiy to predict cases up to three weeks prior.
\item
  \href{https://doi.org/10.1371/journal.pcbi.1004239}{Hickmann et
  al.~(2015)} demonstrate that it is possible to use Wikipedia page view
  statistics and CDC influenza-like illness (ILI) reports to create a
  weekly forecast for seasonal influenza, finding that that Wikipedia
  article access are highly correlated with historical ILI records,
  allowing for highly accurate disease forecasts several weeks before
  case data is available.
\end{itemize}

Given the existing literature, this thesis aims to fill the research gap
by investigating whether Wikipedia can be predictive of mpox cases
during the 2022-2023 multi-country outbreak. While previous studies have
evaluated Wikipedia data for COVID-19 and others have utilized Google
Trends data for mpox, this thesis represents the first attempt, as far
as I am aware, to explore the relationship between Wikipedia page view
statistics and mpox cases.

\hypertarget{research-question}{%
\section{Research Question}\label{research-question}}

My research question is as follows:

\begin{tcolorbox}[enhanced jigsaw, colframe=quarto-callout-note-color-frame, leftrule=.75mm, breakable, opacityback=0, bottomrule=.15mm, left=2mm, rightrule=.15mm, colback=white, toprule=.15mm, arc=.35mm]

To what extent can Wikipedia data be effectively utilized as an
alternative method for measuring public attention and
information-seeking behavior during the 2022-2023 multi-country mpox
outbreak?

\end{tcolorbox}

This research question engages the following current and relevant
conversations within the literature:

\begin{itemize}
\item
  \emph{Open Source Data for Public Health Surveillance}: Examining the
  utility and limitations of using public data sources like Wikipedia to
  monitor and assess public health events, contributing to ongoing
  discussions about their reliability and relevance.
\item
  \emph{Information Dissemination and Public Awareness}: Investigating
  the extent to which public awareness of outbreaks is shaped by the
  impact (number of cases and/or deaths), connecting with debates about
  information ecosystems and their impact on public health
  communication.
\item
  \emph{Policy Implications}: Discussing the potential policy
  recommendations and interventions that can arise from a better
  understanding of public attention and information-seeking behavior
  during outbreaks on digital platforms.
\end{itemize}

My thesis also contains several sub-questions to be investigated in
support of the main research question:

\begin{itemize}
\item
  Which medical articles saw traffic volume increase significantly after
  the start of the mpox outbreak?
\item
  To what extent do the number of mpox cases correlate with the traffic
  volume of mpox-related Wikipedia articles?
\item
  How effective is Wikipedia analytics data compared to other data
  sources (e.g., Google Trends) when it comes to gauging public
  attention towards the mpox outbreak?
\end{itemize}

\hypertarget{data-methods}{%
\section{Data \& Methods}\label{data-methods}}

To answer my research question, I will rely on two main data sources.
First, country-level data on the weekly number of mpox cases is sourced
from WHO.\footnote{\url{https://worldhealthorg.shinyapps.io/mpx_global/}}
Second, Wikipedia analytics data on daily page view volume by article is
sourced directly from Wikipedia.\footnote{\url{https://wikitech.wikimedia.org/wiki/Data_Engineering/Systems/AQS}}
For more details on these data sources, please refer to the
\href{https://rawcdn.githack.com/smkerr/Thesis_WorkInProgress/698d632e8fc01347022f33f5c1837399bba0f25e/proposal/data-report/data-report.html}{Data
Report}.

While mpox case data is available for 117 WHO Member States and
Wikipedia page view statistics exist for nearly 300 languages, I will
limit the scope of this analysis to the 10 countries with the most
cumulative cases, including the United States of America (31,246),
Brazil (10,967), Spain (7,752), France (4,171), Colombia (4,090), Mexico
(4,078), the United Kingdom (3,875), Peru (3,812), Germany (3,800), and
China\footnote{Cases shown include those reported in mainland China
  (1,611), Taiwan (333), Hong Kong (80), and Macao (1).} (2,025).
Accordingly, I will limit the analysis to the Wikipedia projects for the
languages that prominently feature in these countries, including English
(the United States of America and the United Kingdom), Portuguese
(Brazil), Spanish (Spain, Mexico, and Peru), French (France), German
(Germany), Chinese (China). I will examine the time period from 1
January 2022 to 31 December 2023.

\begin{longtable}[]{@{}
  >{\raggedright\arraybackslash}p{(\columnwidth - 4\tabcolsep) * \real{0.3714}}
  >{\raggedright\arraybackslash}p{(\columnwidth - 4\tabcolsep) * \real{0.3000}}
  >{\raggedright\arraybackslash}p{(\columnwidth - 4\tabcolsep) * \real{0.3286}}@{}}
\toprule\noalign{}
\begin{minipage}[b]{\linewidth}\raggedright
\textbf{Country}
\end{minipage} & \begin{minipage}[b]{\linewidth}\raggedright
\textbf{Number of cases}
\end{minipage} & \begin{minipage}[b]{\linewidth}\raggedright
\textbf{Wikipedia project}
\end{minipage} \\
\midrule\noalign{}
\endhead
\bottomrule\noalign{}
\endlastfoot
United States of America & 31,246 & English \\
Brazil & 10,967 & Portuguese \\
Spain & 7,752 & Spanish \\
France & 4,171 & French \\
Colombia & 4,090 & Spanish \\
Mexico & 4,078 & Spanish \\
United Kingdom & 3,875 & English \\
Peru & 3,812 & Spanish \\
Germany & 3,800 & German \\
China & 2,025 & Chinese \\
\end{longtable}

My proposed methodology takes inspiration from th work of
\href{https://doi.org/10.1002\%2Fjmv.28382}{Yan et al.} and
\href{https://doi.org/10.3390/ijerph20043395}{Du et al.} I will conduct
an observational study to assess the lag-correlation between public
attention and mpox cases for the 10 countries with the most cumulative
cases.

\begin{enumerate}
\def\labelenumi{\arabic{enumi}.}
\tightlist
\item
  Define collection of mpox-related Wikipedia articles

  \begin{itemize}
  \tightlist
  \item
    Identify articles directly related to mpox, including historical
    information, symptoms, treatment, and prevention.
  \item
    Identify articles with a low degree of separation within network of
    Wikipedia articles.
  \item
    Analyze Wikipedia page view statistics to identify medical articles
    that experienced significant increases in traffic coinciding with
    the timeline of the 2022-2023 mpox outbreak.\footnote{An assessment
      would need to be made to determine that the increase in traffic
      volume of certain medical articles following the start of the
      outbreak is not spurious but substantial.}
  \end{itemize}
\item
  Data preparation

  \begin{itemize}
  \tightlist
  \item
    Collect daily mpox case numbers from WHO.
  \item
    Extract daily traffic volume data for the defined collection of
    Wikipedia articles using Wikipedia's API with the \{waxer\} package
    using R.
  \item
    De-noise data by aggregating both mpox cases and Wikipedia page view
    statistics to the weekly level.
  \item
    Standardize Wikipedia traffic volumes to be expressed as a
    percentage of total traffic volume for a given Wikipedia language
    version.
  \end{itemize}
\item
  Statistical analysis

  \begin{itemize}
  \tightlist
  \item
    Perform Spearman correlation tests to examine the time-lag
    relationship between Wikipedia traffic volumes and mpox case
    numbers. The range of -21 to +21 days will allow analysis of lead
    and lag effects.
  \item
    Use non-parametric methods, considering the non-normal distribution
    of the data.
  \end{itemize}
\item
  Augmented Dickey-Fuller (ADF) test

  \begin{itemize}
  \tightlist
  \item
    Implement the ADF test to check for stationarity in both the
    Wikipedia traffic and mpox case series. Non-stationary data can lead
    to spurious results in subsequent analyses.
  \end{itemize}
\item
  Vector autoregression (VAR) model

  \begin{itemize}
  \tightlist
  \item
    Develop a VAR model to understand the dynamic relationship between
    the two time series. This model will help in capturing the temporal
    interdependencies and feedback mechanisms between Wikipedia traffic
    and mpox cases.
  \item
    Determine the optimal lag length for the VAR model based on
    information criteria like AIC or BIC.
  \end{itemize}
\item
  Granger causality test

  \begin{itemize}
  \tightlist
  \item
    Apply the Granger causality test within the VAR framework to assess
    whether Wikipedia traffic volumes can be considered a predictor of
    mpox case trajectories.
  \item
    This test will help determine if changes in Wikipedia page views
    precede changes in mpox cases, indicating a predictive relationship.
  \end{itemize}
\item
  Validation and robustness checks

  \begin{itemize}
  \tightlist
  \item
    Conduct sensitivity analyses to test the robustness of the findings
    against different model specifications and subsets of data.
  \item
    Validate the results through comparison with other studies or
    datasets.
  \end{itemize}
\item
  Interpretation and implications

  \begin{itemize}
  \tightlist
  \item
    Interpret the results, while considering the limitations of
    observational data and the potential for confounding factors.
  \item
    Discuss the implications for public health surveillance during
    health emergencies.
  \end{itemize}
\end{enumerate}



\end{document}
